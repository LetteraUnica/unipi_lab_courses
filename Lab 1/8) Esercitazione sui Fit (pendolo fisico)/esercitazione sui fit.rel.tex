\documentclass{article}

\usepackage{adjustbox}
\usepackage{graphicx}
\usepackage{amsmath}
\usepackage{amsfonts}
\usepackage{amssymb}
\usepackage{textcomp}
\usepackage[italian]{babel}

\title{Pendolo semplice}
\author{Lorenzo Cavuoti \\ Alice Longhena}

\begin{document}
	\maketitle
	
	\section{Scopo dell'esperienza}	
	\begin{figure}[!htb]
		\centering
		\includegraphics[width=0.25\textwidth]{/Users/Alicelongh/Documents/LAB/Pendolofisico/Fig_6_7.jpg}
		\caption{pendolo fisico}
	\end{figure}
	
	Utilizzeremo i dati raccolti della precedente esperienza sul pendolo fisico, ovvero le misure dei periodi di oscillazione (nell'approssimazione di piccole oscillazioni) in funzione della distanza dal centro di massa, e vi applicheremo i metodi di fit studiati.
	
	\begin{center}
		\begin{tabular}{cccc}
			\hline
			\multicolumn{4}{c}{Dati raccolti T(d)}\\
			\hline
			\multicolumn{2}{c}{distanza dal centro di massa (m)}& \multicolumn{2}{c}{periodo (s)}\\
			\hline
			\hline
			$47.50$&  $\pm0.15$& $1.635$& $\pm0.005$\\
			\hline
			$37.52$& $\pm0.16$& $1.568$& $\pm0.002$\\
			\hline
			$27.54$& $\pm0.16$& $1.561$& $\pm0.004$\\
			\hline
			$17.55$& $\pm0.17$& $1.680$& $\pm0.001$\\
			\hline
			$7.57$& $\pm0.17$& $2.285$& $\pm0.003$\\
			\hline
		\end{tabular}
	\end{center}
	
	\section{Materiali a disposizione}
		\begin{itemize}
			\item Dati raccolti del pendolo fisico
			\item Calcolatore con ambiente di programmazione python 
		\end{itemize}
	
	\section{Analisi dei dati}
		Come prima cosa abbiamo realizzato tre curve corrispondenti al modello:
		\begin{equation}
			T\left( d \right) = 2\pi\sqrt{\frac{\frac{l^2}{12}+d^2}{gd}}
			\label{modello}
		\end{equation}
		Riferite al valore stimato della lunghezza dell'asta $\hat{l}$ , a $\hat{l}+\sigma_l$ e $\hat{l}-\sigma_l$; nello stesso grafico (figura \ref{3 curve}) sono inseriti i dati relativi a tempi e distanze rilevati sperimentalmente.	
		\begin{figure}[!htb]
			\centering
			\includegraphics[scale=0.70]{/Users/Alicelongh/Documents/LAB/EsercitazionesuiFit/figure_1-1.pdf}
			\label{3 curve}
		\end{figure}
		Al fine di visualizzare la posizione dei punti rispetto alla banda riportiamo gli ingrandimenti relativi a ciascun punto.\\
		\begin{figure}[!htb]
			\centering
			\includegraphics[scale=0.41]{/Users/Alicelongh/Documents/LAB/EsercitazionesuiFit/zoom1.pdf}
		\end{figure}
		\begin{figure}[!htb]
			\centering
			\includegraphics[scale=0.41]{/Users/Alicelongh/Documents/LAB/EsercitazionesuiFit/zoom2.pdf}
		\end{figure}
		\begin{figure}[!htb]
			\centering
			\includegraphics[scale=0.41]{/Users/Alicelongh/Documents/LAB/EsercitazionesuiFit/zoom3.pdf}
		\end{figure}
		\begin{figure}[!htb]
			\centering
			\includegraphics[scale=0.41]{/Users/Alicelongh/Documents/LAB/EsercitazionesuiFit/zoom4.pdf}
		\end{figure}
		\begin{figure}[!htb]
			\centering
			\includegraphics[scale=0.41]{/Users/Alicelongh/Documents/LAB/EsercitazionesuiFit/zoom5.pdf}
		\end{figure}
		Solo il quarto punto � strettamente compreso entro la banda,  l'ultimo e il terzo distano meno di 3 sigma dal modello calcolato nel valore centrale, mente il primo e il secondo distano pi� di 4 sigma dalla media.
		Non risulta quindi un buon accordo tra i dati sperimentali ed il modello.
		\subsection{Fit seguendo il modello}
			Prima di tutto abbiamo verificato l'applicabilit� del fit, ovvero la condizione: 
			\begin{equation}
				|\frac{dT}{dd}(d_i)| \sigma_{d_i} \ll \sigma_{T_i}
				\label{condiz}
			\end{equation} 
			per fare ci� abbiamo ricavato la derivata analiticamente e calcolato il primo membro della disuguaglianza per ogni dato, dal confronto col secondo membro la condizione risulta verificata per il secondo e il terzo punto, ma non per gli altri tre, come si pu� notare nella seguente tabella.
			\begin{tabular}{ccc}
				\centering
				\hline
				$pt$& $| \frac{dT}{dd}| \sigma_{d_i}(m)$& $\sigma_{T_i}(m)$\\
				\hline
				\hline
				$1$& $0.0011$& $0.0053$\\
				$2$& $0.0007$& $0.0022$\\
				$3$& $0.0004$& $0.0048$\\
				$4$& $0.0040$& $0.0013$\\
				$5$& $0.0220$& $0.004$\\					
				\hline
			\end{tabular}
			Trascurando questo inconveniente abbiamo comunque eseguito il fit lasciando l come parametro libero: l risulta $1.049 [m]$, con un errore dell'ordine di $10^{-6}$, il risultato \`{e} quindi compatibile con la misura effettuata con metro a nastro di $1.050 \pm 0.002$ m; il $\chi^2$ risulta $39.2$, molto distante dal valore aspettato di $4.0 \pm 2.8$; infine il p-value risulta $6.4\times10^{-8}$\\\\
			Alternativamente abbiamo calcolato le incertezze efficaci $\sigma_{T_i} \rightarrow \sqrt{{\sigma_{T_i}}^2 +{| \frac{dT}{dd}(d_i)|}^2 {\sigma_{d_i}}^2}$  per i punti che non verificano la condizione (\ref{condiz}), il $\chi^2$ risulta $19.78$, migliore del precedente ma comunque fuori dal valore aspettato, mentre il p-value risulta $5.5\times10^{-3}$.\\\\
			La soluzione migliore risulta eseguire il fit eliminando il secondo punto, in quanto distante pi� di $5 \sigma$ dal modello; cos� facendo il $\chi^2$ risulta $3.9$ e rientra nel nuovo valore aspettato di $3.0 \pm 2.4$.
			\begin{figure}[!htb]
				\centering
				\includegraphics[scale=0.55]{/Users/Alicelongh/Documents/LAB/EsercitazionesuiFit/t_opt-1.pdf}
				\caption{grafico del modello con valore di best fit del parametro l}
			\end{figure}
			\begin{figure}[!htb]
				\centering
				\includegraphics[scale=0.55]{/Users/Alicelongh/Documents/LAB/EsercitazionesuiFit/redidui_t_opt-1.pdf}
				\caption{grafico dei residui rispetto al modello di best fit}
			\end{figure}

	\section{Conclusioni}
	Le misure ricavate sperimentalmente non verificano un buon accordo con i dati, come si pu� notare dal primo grafico nel quale solo un punto \`{e} contentuto entro la banda definita dal modello calcolato in $l+\sigma$ , $l-\sigma$. Tuttavia basta eliminare il secondo punto per ottenere un buon fit ed un parametro di fit in accordo con le misure sperimentali; quindi con alcuni accorgimenti il modello risulta in accordo con i dati sperimentali.

\end{document}







