\documentclass{article}

\usepackage{adjustbox}
\usepackage{graphicx}
\usepackage{amsmath}
\usepackage{amsfonts}
\usepackage{amssymb}
\usepackage[italian]{babel}

\title{Pendolo semplice}
\author{Lorenzo Cavuoti \\ Alice Longhena}

\begin{document}
	\maketitle

	\section{Scopo dell'esperienza}
		Stimare la conducibilità termica di un solido cilindrico, misurando la temperatura in 15 punti a distanza costante
	
	\section{Cenni teorici}
		La quantit\`{a} di calore che si trasmette per conduzione per unit\`{a} di tempo
		\begin{equation}
			W = \frac{dQ}{dt}
		\end{equation}
		\`{e} detta flusso di calore. Una barra in equilibrio termico (se non ci sono perdite) mantiene costante il flusso di calore tra i suoi estremi secondo la legge
		\begin{equation}
			W = -\lambda S\frac{\Delta T}{\Delta x}
			\label{flusso teorico}
		\end{equation}
		Dove $\lambda$ rappresenta la conducibilit\`{a} termica del materiale e $S$ \`{e} la sezione della barra. Ponendo $\Delta x = x_i$ e $\Delta T = T_i - T_0$ la (\ref{flusso teorico}) diventa
		\begin{equation}
			W = -\lambda S\frac{(T_i-T_0)}{x_i}
		\end{equation}
		Ovvero
		\begin{equation}
			T_i = T_0 - \frac{W}{\lambda S}x_i
		\end{equation}
		Quindi ci aspettiamo che la temperatura decresca linearmente al crescere della distanza, con coefficiente angolare
		\begin{equation}
			m=-\frac{W}{\lambda S}
		\end{equation}
		Dove $W = VI$.
		In questo caso il generatore, essendo collegato ad entrambi i cilindri, alimenta due resistenze in parallelo, quindi
		\begin{equation}
			W=\frac{WI}{2}
		\end{equation}
	
	\section{Apparato sperimentale e strumenti}
		Due barre cilindriche dotate di 15 fori, di cui una rivestita di materiale isolante
		Due termistori
		Calcolatore con programma di acquisizione
		Un alimentatore chiuso su due resistenze in parallelo
		Un circuito di acqua corrente
		Metro a nastro (risoluzione 1mm)
	
	\section{Descrizione delle misure}
		Abbiamo misurato la distanza di ciascun foro dall'estremo caldo con il metro a nastro. Successiamente con il programma di acquisizione abbiamo registrato 100 temperature per foro, tenendo fisso un termistore nel primo foro e variando la posizione del secondo (inserire tabella). Da ogni acquisizione abbiamo ricavato media e scarto.
	
	\section{Analisi dei dati}
		Per entrambi i cilindri abbiamo realizzato un grafico in python(figura \ref{fig:cilindro-non-isolato} e  \ref{fig:cilindro-isolato}) con la temperatura sulle ordinate e la distanza dall'estremo riscaldato sulle ascisse. Abbiamo inoltre calcolato la line of best fit con la funzione curve\_fit del modulo scipy di python.\\\\Per il cilindro non isolato risulta
		\begin{equation}
			m = -\frac{W}{\lambda S} = -5.69\pm0.04 [C\:cm^{-1}]
		\end{equation}
		\begin{equation}
			\lambda = 410\pm 60 [W\:m{-1}\:C{-1}]
		\end{equation}
		Il $\chi^2$ della retta purtroppo risulta $1213$ molto pi\`{u} elevato del valore aspettato di $12$.\\\\Per il cilindro isolato abbiamo invece	
		\begin{equation}
			m = -\frac{W}{\lambda S} = -5.88\pm0.004
		\end{equation}
		\begin{equation}
			\lambda = 410\pm 60 [W\:m{-1}\:C{-1}]
		\end{equation}
		Il $\chi^2$ della retta anche in questo caso \`{e} elevato, risultando $567$ lontano dal valore aspettato di $12$.
	
	\begin{figure}
		\centering
		\includegraphics[width=0.7\linewidth]{"Cilindro non isolato"}
		\caption{Grafico temperatura in funzione della distanza}
		\label{fig:cilindro-non-isolato}
	\end{figure}
	\begin{figure}
		\centering
		\includegraphics[width=0.7\linewidth]{"Cilindro isolato"}
		\caption{Grafico temperatura in funzione della distanza}
		\label{fig:cilindro-isolato}
	\end{figure}
	
	\section{Conclusioni}
		La conducibilità termica λ del materiale \`{e} direttamente proporzionale al flusso di calore W attraverso la barra. Essendo W più alto nel caso isolato, la conducibilit\`{a} associata a quest'ultimo dovrebbe risultare più alta.
		Tuttavia i nostri dati mostrano il contrario, tale contraddizione potrebbe essere dovuta al fatto che la temperatura del sistema \`{e} molto sensibile dalla temperatura dell'ambiente esterno. Inoltre avendo effettuato le misurazioni relative al cilindro isolato successivamente a quelle relative al cilindro non isolato le condizioni dell'esperienza potrebbero essere state differenti. La conducibilità in entrambi i casi si avvicina a quella riconosciuta per il rame di circa 400 W/mC. Il $\chi^2$ risulta in entrambi i casi troppo grande, probabilmente si deve aumentare l'errore associato alla temperatura in quanto non si tiene conto della variazione di temperatura esterna durante il tempo impiegato per le misure
		
\end{document}